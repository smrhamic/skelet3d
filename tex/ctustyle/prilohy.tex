
\bibchap
\ifx\usebib\undefined
   \usebbl/c mybase
\else
   \usebib/c (simple) mybase
\fi

\label[zadani]
\app Zadání práce

Tento dokument specifikuje šablony pro \LaTeX{} a MS Word, které jsou
doporučeny pro psaní bakalářských, diplomových nebo disertačních prací
na ČVUT FEL. Specifikace se opírá
o~dokumenty~\cite[CVUT-FEL:zaverecne-prace, CVUT-FEL:smernice, CSN:016910].

\bigskip

\noindent
Šablony mají splňovat následující požadavky:
\begitems
* Písmo Latin Modern (v~\LaTeX{} instalacích je standardně
  obsaženo, pro MS Word bude OTF verze s~podporou matematiky přiložená
  k~šabloně). Velikost základního písma 11 bodů.
* Implicitní kódování šablon UTF-8.
* Formátování na papír A4, vnitřní okraj 30\,mm pro pevnou
  vazbu, délka řádky přizpůsobena velikosti písma.
* Implicitně se předpokládá oboustranná sazba.
* Strukturní elementy: titulní list, poděkování, prohlášení,
  abstrakt + klíčová slova (cz/en), obsah, seznam symbolů/zkratek,
  přílohy, bibliografie, tabulky a obrázky s~popisky.
* Číslování stránek od 1. strany textu (úvodu); úvodní stránky
  číslovány římsky. Důvodem je snadno rozpoznatelný rozsah práce.
* V~záhlaví stránky číslo a název hlavní kapitoly. V~patičce
  u~vnějšího okraje číslo stránky.
* Součástí šablony bude styl pro bibliografie s~číselnými odkazy;
  v~seznamu literatury, řazení dle pořadí citování.
* Šablona umožní následující varianty výsledného dokumentu:
  \begitems \style x
    * bakalářská/diplomová/disertační práce (předpokládá se stejná
      základní struktura, jen změna podtitulků),
    * anglický nebo český jazyk textu (vzory dělení, nadpisy,
      číslování kapitol),
    * pracovní verze (draft) s~textem ,,Draft +
      datum'' v~patičce.
  \enditems
\enditems

\label[zkratky]
\app Zkratky a symboly

\font\mflogo=logo10 at11pt
\def\METAFONT{{\mflogo METAFONT}}
\def\METAPOST{{\mflogo METAPOST}}

Tento text je až na výjimky převzat z~\cite[zyka].

\sec Zkratky

Jako příklad pro popis zkratek poslouží pojmy ze světa \TeX{}u.

\medskip
\bgroup \leftskip=6.3em
\abbrv[\TeX{}]  Program na přípravu elektronické sazby vysoké kvality
   vytvořený Donaldem Knuthem. Program zahrnuje interpret makrojazyka.
   Název programu se vyslovuje \uv{tech}.
\abbrv[\METAFONT{}] Program a makro jazyk pro generování fontů
   z vektorového do bitmapvého formátu vytvořený Donaldem Knuthem.
\abbrv[\METAPOST{}] Program generující vektorovou grafiku založený na
   \METAFONT{}u vytvořený Johnem Hobby.
\abbrv[plain\TeX{}]  Originální \TeX{}ový formát (rošiření na úrovni 
   makrojazyka). Je součástí každé distribuce \TeX{}u a je
   vytvořen Donaldem Knuthem.
\abbrv[\csplain{}] \TeX{}ový formát rozšiřující plain\TeX{} o možnosti sazby
   v českém a slovenském jazyce vytvořený Petrem Olšákem.
\abbrv[\LaTeX{}]  Nejznámější \TeX{}ový formát (rozšíření na úrovni 
   makrojazyka) vytvořený Leslie Lamportem. 
   Existuje obludné množství různých balíčků, které pomocí
   makrojazyka \TeX{}u dále rozšiřují výchozí možnosti \LaTeX{}u.
   Rozličné uživatelské požadavky jsou nejčastěji řešeny použitím vhodného balíčku.
\abbrv[OPmac] Olšákovy Plain\TeX{}ová makra nabízející uživatelům
   plain\TeX{}u podobné možnosti, jako \LaTeX{}, ovšem přímočařeji
   a jednodušeji.
\abbrv[Con\TeX{}t]  Typografický systém vystavěný na rozšíření \TeX{}u s
   názvem Lua\TeX{} (kombinuje makrojazyk \TeX{}u s jazykem Lua) a na množství
   předpřipravených makro souborů vytvořený týmem v čele s Hansem Hagenem.
   Rozličné uživatelské požadavky jsou nastavovány pomocí přiřazení hodnot
   klíčovým slovům společně s možností \TeX{}ového, \METAPOST{}ího a 
   Lua programování.
\par\egroup


\sec Symboly

\medskip
\bgroup \leftskip=2em
\abbrv[$\pi$] Konečná verze \TeX{}u zmíněna v Knuthově \TeX{}tamentu.
\abbrv[e] Konečná verze \METAFONT{}u.
\abbrv[$2\varepsilon$] Současná verze \LaTeX{}u používaná od roku 1994.
Počítá se s ní jako s přechodnou verzí mezi původní Lamportovou 
verzí \LaTeX{}~2.09 a cílovou verzí \LaTeX{}~3. Tento přechodný stav už trvá
19~let.
\par\egroup


