We have successfully developed an online application called {\it Skelet3D}, an editable database of bones including labeled 3D models: We researched existing solutions, identified improvements, designed, implemented, tested and deployed the application.

\sec Summary

First we defined the requirements and key features of the web application: A hierarchy of categories and subcategories containing pages related to bones, each page containing a 3D model with labels, text and images. Labels and text need to be editable while models and images can be uploaded into the system. The application should be bilingual.

We have performed a review of existing solutions and this project’s predecessor, finding that none of the solutions cover this project’s utility. It was decided that the previous version had too many issues limiting future development and that this project should start with a clean slate.

Available technologies were researched and compared and after careful consideration, {\tt three.js} framework using \glref{WebGL} was chosen for 3D rendering, while the server would run easily extensible Java \glref{EE} using \glref{JSF} -- an \glref{MVC} framework.

We designed the application: Identified all its use cases and assembled their step-by-step scenarios. Based on this, a mockup \glref{GUI} covering these scenarios was drafted. Business objects figuring in the application were identified and a database schema supporting multilingual content was created. The internal structure of the application was then modeled as a layered application based on \glref{MVC} architecture.

According to this design, the application was implemented. All layers including database persistence, data-representing entity classes, service classes with business logic, backing beans and their respective \glref{HTML} views, including client side JavaScript with 3D, were successfully implemented. All important features were realized: Categories and pages can be browsed as a hierarchy or searched for directly, models can be viewed from all angles, several modes of displaying labels are available and labels can be selected. All interfaces are bilingual; content may be created in either language. Authenticated users can create and edit pages, their content and labels, upload models and images. Content from previous version has been transferred.

The application was deployed to a production server in December 2014.

Functionality was verified by walking through individual use case scenarios -- all features are available except for account administration. The application is confirmed to work in \glref{WebGL} compatible browsers on desktop devices as well as more recent mobile devices including Android phones.

During later stages of development we cooperated with content editors, who provided useful feedback and helped improve user experience before public release. Once they had input enough example content, the application was presented to medical students, ten of whom submitted their opinions in a short survey. The feedback was largely positive, praising the application’s utility.

\sec Future Development

While the application is working as intended and users seem to be satisfied, there is always room for improvement.

Firstly, the account management system is still missing, which means user accounts are managed directly through the database. An interface allowing administrators to modify editors’ accounts should be implemented, as well as an option to change passwords anytime.

Secondly, better mobile support could be provided. While the number of devices supporting \glref{WebGL} by default can be expected to grow, there should be a guide regarding compatibility and what steps can be taken to enable \glref{WebGL} on older devices. It might also be worth revising the touch controls of 3D, which work, but sometimes lead to conflicts between controlling the scene and controlling the page.

Other future improvements may include:

\begitems
*Conversion of models to a more efficient internal format, which would both save bandwidth and allow smooth shading
*Improvement of the “Search” function to search all content (including 3D labels) rather than just page/bone names
*Division of page content into chapters (allows table of content, collapsible sections)
*Better tools for handling citations/sources, which now need to be created manually
*Personal notes and/or public comment section for registered users
*Overall improvement of graphical design
\enditems

Once obvious improvements are implemented and the content is more refined, user feedback should be collected in order to find out what needs to be improved further.

\sec Goal Completion Summary

All tasks of the assignment were successfully completed.

\begitems
*Current solution and other existing solutions were reviewed
*Improvements and new key features were introduced
*A new version of the application was designed
*The application was implemented according to the design
*Functionality was verified and tested
*User experience was tested with editors and students
*The application was improved according to user feedback
\enditems