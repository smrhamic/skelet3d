In this chapter we define the requirements and features of the application, compare the basic concept to existing solutions including the previous version of the Atlas and evaluate possible contributions of the project.

\sec Role Definition

The application will support three basic user roles:

\begitems
*{\bf User} -- Any visitor of the website. Users can browse all public content but not alter it in any way.
*{\bf Editor} -- An authenticated user responsible for uploading and editing the content.
*{\bf Administrator} -- An authenticated user responsible for managing user accounts.
\enditems

\sec Requirements

\secc Functional Requirements -- User

\vskip 10pt
{\bf 3D Viewer with following features:}

\begitems
*Display 3D models of bones
*Display labels for significant parts of the bones
*Display detailed description of a part when label is selected
*3 display modes for labels: Full labels / Labels with hidden names / No labels
*Adjustable view (move, rotate, zoom)
\enditems

\vskip 10pt
{\bf Website with following features:}

\begitems
*Navigation to specific bones using a hierarchy of categories
*Navigation using a "Search" function
*Web page for each bone with a 3D model, relevant images, text and references
\enditems

\secc Functional Requirements -- Editor

\begitems
*Login into editorial system
*Manually add, edit and delete pages (bones)
*Upload 3D model files
*Upload image files
*Add uploaded 3D models and images to pages
*Manually add, edit and delete textual information related to each bone
*Manually add, edit and delete labels and descriptions of 3D models
\enditems

\secc Functional Requirements -- Administrator

\begitems
*Login into account management system
*Create, edit or delete editor accounts
\enditems

\secc Non-functional Requirements

\begitems
*Browser-based online application
*Multiplatform
*Support for bilingual content (Czech, English)
\enditems

\sec Target Group

The application is targeted at students of the Third Faculty of Medicine at Charles University in Prague. Other users are to be expected, but the focus should remain on medical students.

Medical students are young, non-technical users. However, high working knowledge with a web browser can be expected.

\sec Existing Solutions and Alternatives

The following list is by no means complete. These are just several examples of the most relevant solutions and alternatives found online.

\begitems
*Skelet 3D (available at \cite[skelet3d] {\it Note: No longer available as of December 2014.}) is the predecessor of our project, further discussed in \ref[old_atlas]. Our intention is to redesign and upgrade this solution.

*Skeletopedia (available at \cite[skeletopedia]) seems to be the independent solution closest to our specification. It provides interactive 3D models of bones in browser environment as well as simple labeling of certain parts, unfortunately not in great detail. The contribution of our project in comparison to this solution should be more detailed and far more numerous labeling of the models as well as further information about each label and each bone in general.

*Zygote Body, Anatronica (available at \cite[zygote, anatronica]) and many other systems include 3D models of the human skeleton. However, they provide little to no additional information besides the name of each bone.

*3D Science provides extremely detailed 3D models (available at \cite[3dscience]). However, these are not free, most likely not suitable for online display and most importantly provide no additional information.

*Palacký University in Olomouc provides information on most bones in their online human anatomy as well as photography of human bones, sometimes including labels with short descriptions (available at \cite[upol]). The information, however, seems incomplete. Moreover, there is no 3D material available, which is the key feature of our project.

*Wikipedia provides rather detailed information on human bones (available at \cite[wikibones]). English version includes simple 3D animations of most bones, but it doesn’t provide any connection of 3D models with labels and descriptions, which is the intended contribution of our project.

\enditems

Overall, each existing solution provides some interesting features, but none of them offers everything our project is aiming for. We will incorporate the best of each existing solution and add something extra: Interactive 3D models comparable to Zygote/Anatronica, labels more detailed than those of Skeletopedia, information comparable to Wikipedia but focused on medical students and up-to-date content in Czech.

\label[old_atlas]
\sec Analysis of the Current Version of the Atlas

The Atlas project is not beginning with this thesis. It was initiated in 2012 and most of the development of the existing version happened in 2013.

The application was developed on the fly and went through a series of overhauls without proper planning. Although it works, it suffers from a variety of flaws. From a developer's point of view, the internal structure is unorganized and undocumented. While it might had been sufficient, the lack of order and proper organization makes sustainability and further development a complicated issue.

The current version is a publicly available website running on the servers of the Third Faculty of Medicine, Charles University \cite[skelet3d] {\it Note: In December 2014 that version was replaced by the product of this project and is no longer available.}

It consists of a list of bones sorted in a hierarchy of groups based on parts of the human body, a 3D viewer of models of bones and a very simple editing page which isn’t public. An as-of-yet unreleased bilingual version allows editing in Czech and English (as opposed to Czech only).


\secc Bone Selector

The list of available bones is a Flash application written in ActionScript 3 that allows selection of a model and filtering using a hierarchy of predefined groups. It does what it’s supposed to do, although the layout might not be the most intuitive and there are no additional features such as “search”. The loading time is longer than expected due to inefficient use of \glref{XML} files that define groups and bones.

\secc 3D Viewer

The model viewer is another Flash (ActionScript 3) application using a simple 3D engine called Sandy with two basic modes: view and edit. It allows the user to manipulate view freely, although not in the most intuitive way, utilizing only one mouse button. It allows users to display or hide labels and pins appropriately.

However, it suffers from graphical errors, most notably seeing labels through a bone while they should be behind it and vice versa. This is caused by Z-sorting algorithms used in Sandy 3D engine, which are too simple for the task.

It is also missing some minor tweaks such as adjusting the label width to fit the length of the text.

The edit mode is working, although its graphical design is poor.

The rendering speed of the application is unsatisfactory, greatly limiting the use on mobile devices and the quality of models. Some of the more complex models are reaching low \glref{FPS} even on average desktop computers, while being virtually unusable on older machines and mobile devices. A tradeoff between model complexity and rendering speed is to be expected, but current application’s performance is nevertheless underwhelming. The technology used doesn’t provide easy solutions to some of the identified flaws (see \ref[flaws]).

\secc PHP Parts

The Flash application is nested in a simple \glref{PHP} page. The web page which is used to upload new models and enter the editing mode is implemented as a series of simple \glref{PHP} scripts. Other \glref{PHP} scripts are called by the Flash application to handle \glref{XML} files on the server.

\secc Editing

There are two versions of the application, a public version with editing disabled and a separate version for editing only, whose location is not known to public, but which is not protected in any other way. This {\it security by obscurity} is absolutely insufficient, so to prevent vandalism, the live version is updated manually by transferring data files from the editable version. Introduction of a login system would allow better sustainability through direct editing by privileged users as opposed to manual transfers by the website administrator.

\secc Identified Flaws

\clabel[flaws]{Identified Flaws}

{\def\sizespec{at9pt} \resizeall \tenrm
\ctable{llclc}{
\hfil Itendified Flaw & Affects & Importance & Suggested Solution & Difficulty\crl \tskip4pt
No images / text & Content value & High & Add editorial system & High \cr
No authentication & Safety, sustainability & High & Add login system & Medium \cr
Hardware demands & Availability, details & High & Change of 3D engine & High \cr
Graphical errors & User experience & Medium & Change of 3D engine & High \cr
Imperfect \glref{GUI} & User experience & Medium & Rework \glref{GUI} & Medium \cr
Bad internal design & Future development & Medium & Proper design & Medium \cr
No documentation & Future development & Medium & Add documentation & Low \cr
No search & User experience & Low & Implement search & Low \cr
Slow loading of lists & User experience & Low & Use \glref{DB} over \glref{XML} & Low \cr
}
}
\caption/t Identified flaws of the current version of the Atlas. Author's subjective evaluation of their impact and solutions.


\secc Conclusion

Overall, the technologies used seem to be inefficient and partly outdated. A complete reworking of the system seems to be the best solution considering the extent of individual improvements, especially the change of 3D engine and introduction of an editorial system.

The models and labels entered by medical students into the existing version should be valid and reusable in the new implementation.

\sec Project's Contributions

The main goal is to help medical students acquire knowledge of human bones. There are numerous sources available, ranging from lectures and printed textbooks to aforementioned online solutions. Our application cannot compete with the experience and insights of lecturers and it is not likely to replace textbooks because of the sheer volume of information included in those. But it can be a great study material, available anytime, hopefully better than other online sources.

None of the known sources (other than the previous iteration of our Atlas) provide what we hope to achieve in this project: Exhaustive information linked to illustrative 3D models. There are 3D models and there is information. This project’s contribution is linking the two in an interactive manner, allowing students easy navigation and a mental link between spatial and textual information.

This was attempted in the current version of the Atlas in a clumsy and imperfect manner. This version strives to be a more professional solution of the problem without all the flaws of the last version.

The aim is to develop the application in an organized, orderly and documented way, using the best available technologies and methods chosen after careful consideration. We need to avoid the glitches and imperfections of the previous iteration, to make the display module more efficient, the interface more user-friendly. We need sustainability, most notably a proper editorial system with authentication to allow the application to grow and fill up with useful information.

Moreover, there will be the new feature of web pages with additional information. They will contain any information seen fit by the editors, giving this project the potential to be a truly exhaustive source of knowledge regarding human bones, ranging from anatomy to pathology. Other new features might be added as well, given a time reserve.