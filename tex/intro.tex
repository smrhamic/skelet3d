The topic of this thesis is the development of the Atlas of Human Bones, an online database of bones including 3D models with labels and web pages with additional information. The Atlas will serve as a source of information and a study material, primarily for students of the Third Faculty of Medicine at Charles University in Prague.

\sec Project Background

The Third Faculty of Medicine at Charles University in Prague is a successful, established educational institution. Its Department of Anatomy, as many others before, have realized the great potential of computer technologies in education and started numerous projects to transfer knowledge from traditional printed sources to a much more flexible online domain. One of these projects was an online atlas of human bones.

This project was initiated in 2012 by doc. MUDr. Václav Báča, Ph.D., at the time a faculty member of the Department of Anatomy. The project aimed to create an online 3D database of human bones, which would be an exhaustive yet illustrative source of information for medical students. The goal was not only to display 3D models, but also to provide numerous labels with descriptions of relevant structures of each bone.

I developed the first, rather imperfect (see \ref[old_atlas]) version of the Atlas in 2013 and the Third Faculty of Medicine provided the content.

Working on this project as a part of my master’s thesis finally gave me the necessary amount of time to give the Atlas a long due redesign.

\sec Thesis Content

In this thesis we first define the features of the Atlas, compare the concept with existing solutions to see if it brings anything new and analyze the existing version to see if a complete redesign is necessary. We research possible technologies and choose those best suited for various parts of our application. We design a sustainable web application whose content is not only accessible to medical students, but also editable by users without a technical background. We describe the actual implementation of the design, acquisition of content and deployment to a live server. Finally we collect feedback from potential users to see what needs to be improved in future versions.