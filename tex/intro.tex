\sec Project Aims

The aim of this project is to help medical students learn about bone anatomy and pathology. Traditional printed sources provide detailed information, but the 2D form makes visualization difficult. An alternative is studying real bones available at the institute of anatomy, which seems extremely inconvenient.

In order to make students’ lives easier, we intend to develop the Atlas of Human Bones, a source of information and a study material, primarily for students of the Third Faculty of Medicine at Charles University in Prague. The Atlas will be an online database of bones including 3D models and web pages with additional information. The models will also include detailed labels -- name tags for important structures of each bone, which will provide further details when selected.

Presentation of the content is only one part of the application though, as the content needs to be editable by medical experts without a technical background.

Our first task before we can begin development is to define the key features of our application. Then we need to perform a review of existing solutions to make sure our design is advantageous and would be beneficial for medical students (not only those of the Third Faculty of Medicine). Since this project has a predecessor implementing some of the features, it needs to be analyzed to find out whether it can be upgraded or a complete redesign is necessary.

The next step is researching available technologies and choosing those best suited for various parts of our application.

Then we must properly design a sustainable web application, which requires identifying the application’s use cases, identifying relevant business objects and creating a data model, designing a user interface and modeling the internal structure of the application.

Once the design is ready, we need to implement it.

After the implementation is complete and its functionality verified, it should be deployed to a server and tested with regular users as well as editors. The application may then be improved according to user feedback if time permits.


\sec Project Background

The Third Faculty of Medicine at Charles University in Prague is a successful, established educational institution. Its Department of Anatomy, as many others before, have realized the great potential of computer technologies in education and started numerous projects to transfer knowledge from traditional printed sources to a much more flexible online domain. One of these projects was an online atlas of human bones.

This project was initiated in 2012 by doc. MUDr. Václav Báča, Ph.D., at the time a faculty member of the Department of Anatomy. The project aimed to create an online 3D database of human bones, which would be an exhaustive yet illustrative source of information for medical students. The goal was not only to display 3D models, but also to provide numerous labels with descriptions of relevant points of each bone. That is of utmost importance as medical students are not only interested in the name of the bone, but also its  significant structures and its attachment to neighboring muscles and bones by tendons and ligaments.

I developed the first, rather imperfect (see \ref[old_atlas]) version of the Atlas in 2013 and the Third Faculty of Medicine provided the content.

Working on this project as a part of my master’s thesis finally gave me the necessary amount of time to give the Atlas a long due redesign.

\sec Goal Summary

\begitems
*Review the current solution and other existing solutions
*Identify possible improvements
*Design a new version of the application
*Implement the application according to the design
*Verify and test functionality
*Test user experience with editors and students
*Improve application according to user feedback
\enditems